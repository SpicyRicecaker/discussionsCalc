\documentclass[letterpaper, 12pt]{article}
\usepackage[margin=1in]{geometry}
\usepackage{amsmath}
\usepackage{amssymb}
\usepackage{fancyhdr}
\usepackage{hyperref}
\usepackage{xcolor}
\setlength{\headheight}{15pt}

\pagestyle{fancy}
\fancyhf{}

\rhead{
    Shengdong Li
    Calc 1
}
\rfoot{
    Page \thepage
}

\usepackage{indentfirst}

\begin{document}
\title{Response to Sashank Meka}
\author{by Shengdong Li}
\date{16 May 2020}
\maketitle

\section{Intro}
Hey Sashank! Good job on the initial lesson. I really enjoyed your explanations for each of your steps and the work you showed, like which half angle identity you used. However, I did notice another way that you could've approached the first problem and a few other small complaints.

\section{Regarding Creative Substitution}
So to solve for the $A$ and $B$ values in the partial fraction decomp, it looked like you chose to solve it using a system of equations. Another way that you could've approached this problem is by substituting out some values for $x$ that might get you the $A$ and $B$ values. And for this specific problem, the linear factors make creative substitution really easy.
\begin{align}
    \intertext{At this step right here...}
    \frac{x+5}{\left(x+2\right)\left(x-1\right)}&=\frac{A}{x+2}+\frac{B}{x-1}
    \intertext{After you multiply out the denominator of the fraction to the left...}
    x+5&=A\left(x-1\right)+B\left(x+2\right)
    \intertext{Instead of distributin factors and solving it as a system of equations, you can set $X$ to some value that cancels out a variable. Setting $x=1$ in this case removes the $A$ and gives us the value of $B$}
    x&=1                                                                               \\
    \left(1\right)+5&=A\left(\left(1\right)-1\right)+B\left(\left(1\right)+2\right)    \\
    6&=3B                                                                              \\
    B&=2
    \intertext{While setting $x=-2$ removes the value of $B$ and gives us the value of $A$}
    x&=-2                                                                              \\
    \left(-2\right)+5&=A\left(\left(-2\right)-1\right)+B\left(\left(-2\right)+2\right) \\
    3&=-3A                                                                             \\
    A&=-1
    \intertext{From there, you can just solve the integral normally.}
\int_{ }^{ }\frac{x+5}{\left(x+2\right)\left(x-1\right)}dx&=\int_{ }^{ }\frac{-1}{x+2}+\frac{2}{x-1}dx\\
&=2\int_{ }^{ }\frac{1}{x-1}dx-\int_{ }^{ }\frac{1}{x+2}dx\\
&=\boxed{2\ln\left|x-1\right|-\ln\left|x+2\right|+C}
\end{align}
\section{Other Small Things}
Don't forget that $\ln$ s have absolute values, and always add the $+C$ to the final integral!
\section{Conclusion}
Overall, I think that you did a great job solving both integrals and explaining how you did it. The small $+C$ and $\ln$ errors were the only ones I could find. Good job Sashank!
\end{document}
