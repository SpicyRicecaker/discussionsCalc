\documentclass[letterpaper, 12pt]{article}
\usepackage[margin=1in]{geometry}
\usepackage{amsmath}
\usepackage{amssymb}
\usepackage{fancyhdr}
\usepackage{hyperref}
\usepackage{xcolor}
\setlength{\headheight}{15pt}

\pagestyle{fancy}
\fancyhf{}

\rhead{
    shengdong li
    calc 1
}
\rfoot{
    page \thepage
}

\usepackage{indentfirst}

\begin{document}
\title{Response to Camden Maggard}
\author{by Shengdong Li}
\date{17 May 2020}
\maketitle

\section{Intro}
Hey Camden! I'm glad to hear that you liked my lesson, and thought that the presentation was professional. 
\section{Similarity of Questions Compared}
In regards to the questions, thank you for saying that you thought they were pretty similar. I chose these two questions specifically because I found it really interesting that while there were differences in the $x$ values being in the numerator and the denominator, they both ended up to be trig-sub questions that eventually also required some trig identities. \par
For example, in solving the integral $\int_{ }^{ }\frac{x^{2}}{\sqrt{4-x^{2}}}dx$, we had to trig-sub $x=2\sin\theta$, which, after cancelling everything, resulted in $\int_{ }^{ }\sin^{2}\theta d\theta$. To progress from here, we had to use the half angle identity of $\sin^{2}\theta=\frac{1-\cos\left(2\theta\right)}{2}$, which left us with another integral of $\int_{ }^{ }\sin\left(2\theta\right)d\theta$, which required a double-angle identity, $\sin\left(2\theta\right)=2\sin\theta\cos\theta$ to solve the rest of it. \par
In solving $\int_{ }^{ }\frac{1}{x^{3}\sqrt{x^{2}-9}}dx$, we had to trig-sub $x=3\sec\theta$, which gave us $\int_{ }^{ }\cos^{2}\theta d\theta$, in which we had to use a similar half-angle identity of $\cos^{2}\theta=\frac{1+\cos\left(2\theta\right)}{2}$, that left us with $\int_{ }^{ }\sin\left(2\theta\right)d\theta$ in which the double-angle $\sin$ identity had to be used again: $\sin\left(2\theta\right)=2\sin\theta\cos\theta$.
\section{Conclusion}
In the end, solving these two problems made me realize how important the double-angle and half-angle trig identities were, and let me recognize some patterns of when to use each. Thanks for responding to my initial post Camden!
\end{document}