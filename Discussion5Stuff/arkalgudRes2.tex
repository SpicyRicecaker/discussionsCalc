\documentclass[letterpaper, 12pt]{article}
\usepackage[margin=1in]{geometry}
\usepackage{amsmath}
\usepackage{amssymb}
\usepackage{fancyhdr}

\pagestyle{fancy}
\fancyhf{}
\rhead{
    Shendong Li
    Calc 1
}
\rfoot{
    Page \thepage
}

\begin{document}
\title{(2nd) Response to Aadhya Arkalgud}
\author{by Shengdong Li}
\date{3 May 2020}
\maketitle
\section{Intro}
Hey Arkalgud! Nice catch that I didn't multiply by the reciprocal! I'm very sad that this small error led me to an answer that differed slightly from the right one.

\begin{align}
\intertext{Looking at step 19, if instead of doing $\frac{65}{64}$...}
\frac{65}{64}\left(\frac{1}{2}e^{2x}\cos\left(\frac{1}{4}x\right)+\frac{1}{16}e^{2x}\sin\left(\frac{1}{4}x\right)\right)
\intertext{But $\frac{64}{65}$, the actual reciprocal instead,}
\frac{64}{65}\left(\frac{1}{2}e^{2x}\cos\left(\frac{1}{4}x\right)+\frac{1}{16}e^{2x}\sin\left(\frac{1}{4}x\right)\right)
\intertext{We would be left with $64$ as the numerator,}
\frac{64}{65\cdot2}e^{2x}\cos\left(\frac{1}{4}x\right)+\frac{64}{65\cdot16}e^{2x}\sin\left(\frac{1}{4}x\right)
\intertext{Which would cancel out, leaving me with the same answer that you got.}
\boxed{\frac{32}{65}e^{2x}\cos\left(\frac{1}{4}x\right)+\frac{4}{65}e^{2x}\sin\left(\frac{1}{4}x\right)+C}
\end{align}

\section{Conclusion}
Really nice process that you showed in solving the problem. This reminded me again that I should always double, triple check my work when doing simple calculations. Thank you for the reply!
\end{document}