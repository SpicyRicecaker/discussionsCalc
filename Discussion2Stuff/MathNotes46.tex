\documentclass[letterpaper, 12pt]{article}
\usepackage[margin=1in]{geometry}
\usepackage{amsmath}
\usepackage{amssymb}
\usepackage{fancyhdr}
\usepackage{pgfplots}
\pgfplotsset{compat=1.17}
\usepackage{tikz}

\pagestyle{fancy}
\fancyhf{}

\rhead{
    Shengdong Li
    Calc 1
    Notes
}

\rfoot{Page \thepage}
\usepackage{indentfirst}
\setlength{\parindent}{2em}

\begin{document}
\title{Week 4/6 Notes}
\author{by Shengdong Li}
\date{6 April 2020}
\maketitle

\section{Finding Volume: Fixed base with Cross-Sections}
\begin{enumerate}
    \item Volume problems, with flat base, set of geometric cross-sections
    \item 3D shapes where you need to determine the cross section
\end{enumerate}

\section{Volume Problems}
You have a flat base with changing height, base on the shape in question or information given \par
For example, consider the post-it problem. The cross section = $s\cdot s=s^2=$ area of postit. Let $a(x)$ = area of the $2d$ cross section. The $volume$ of the $3d$ solid would thus be $\int_{a}^{b} A(x)\:dx$. $[a,b]$ is the $x$ interval that forms the structural base. $a(x)$ is just $s^2, \pi r^2$, etc, but must be in terms of $x$. This can be $dy$ or $dx$ \paragraph{own solutions}
Ok, so if $a(x)$ is the area of the $2d$ shape / cross section, and the $3d$ volume is just the integral of that, then the only hard part is putting the side lengths or radius in terms of $x$, in which the base of the shape is bound by. \paragraph{green}
Semicircle equation = $\frac{\pi r^2}{2}$.
And $r = x^2$.Therefore, $A(x)=\frac{\pi (x^2)^2}{2}$. $a=0, b=1$, in terms of $dx$. Integral equation = $\int_{0}^{1}\frac{\pi x^4}{2}\:dx$=$\frac{\pi}{2}\int_{0}^{1}x^4\:dx$=$\frac{\pi}{2}\cdot\frac{x^5}{5}\big|_{0}^{1}$=$\frac{1}{5}\cdot\frac{\pi}{2}$=$\frac{\pi}{10}$ \paragraph{green correction}
$r= \frac{x^2}{2}$. Therefore, $A(x)=\frac{\pi (\frac{x^2}{2})^2}{2}$. $a=0, b=1$, in terms of $dx$. Integral equation = $\int_{0}^{1}\frac{\pi x^4}{8}\:dx$=$\frac{\pi}{8}\int_{0}^{1}x^4\:dx$=$\frac{\pi}{8}\cdot\frac{x^5}{5}\big|_{0}^{1}$=$\frac{1}{5}\cdot\frac{\pi}{8}$=$\frac{\pi}{40}$ \paragraph{blue}
Rectangle with sides of base $s$, height $\frac{s}{2}$, has $A(x)=\frac{s^2}{2}$. In terms of $x$, this should still be $A(x)=\frac{x^2}{2}$. $a=0, b=1$, therefore the integral expression = $\int_{0}^{1}\frac{x^2}{2}\:dx=\frac{1}{2}\int_{0}^{1}x^2\:dx$. This can be rewritten to $\frac{1}{2}\cdot\frac{x^3}{3}\big|_{0}^{1}$. Which is $\frac{1}{6}$
\paragraph{blue correction}
Rectangle with sides of base $s$, height $\frac{s}{2}$, has $A(x)=\frac{s^2}{2}$. If $s=x^2$, this should then be $A(x)=\frac{x^4}{2}$. $a=0, b=1$, therefore the integral expression = $\int_{0}^{1}\frac{x^4}{2}\:dx=\frac{1}{2}\int_{0}^{1}x^4\:dx$. This can be rewritten to $\frac{1}{2}\cdot\frac{x^5}{5}\big|_{0}^{1}$. Which is $\frac{1}{10}$
\paragraph{purple}
Square with sides of base $s$, has $A(x)=s^2$. If $s=x^2$, this should then be $A(x)=x^4$. $a=0, b=1$, therefore the integral expression = $\int_{0}^{1}x^4\:dx=\int_{0}^{1}x^4\:dx$. This can be rewritten to $\frac{x^5}{5}\big|_{0}^{1}$. Which is $\frac{1}{5}$
\paragraph{yellow}
Isoceles triangle with sides of base $s$, has $A(x)=\frac{s^2}{2}$. If $s=x^2$, this should then be $A(x)=\frac{x^4}{2}$. $a=0, b=1$, therefore the integral expression = $\int_{0}^{1}\frac{x^4}{2}\:dx=\frac{1}{2}\int_{0}^{1}x^4\:dx$. This can be rewritten to $\frac{1}{2}\cdot\frac{x^5}{5}\big|_{0}^{1}$. Which is $\frac{1}{10}$
\section{Practice Problem 1}
Find the volume of the solid whose base is bounded by the graphs of $y=x+1$ and $y=x^2-1$, with the indicated cross-sections taken perpendicular to the $x$-axis (vertical).
\begin{center}
    \begin{tikzpicture}
        \begin{axis}[
                axis lines = left,
                xmin=-3,xmax=3,ymin=-2,ymax=4,
                legend pos=outer north east
            ]
            \addplot [mark=none, color=red, samples=100]{x+1};
            \addplot [mark=none, color=blue, samples=100]{x^2-1};
            \legend{$y=x+1$, $y=x^2-1$}
        \end{axis}
    \end{tikzpicture}
\end{center}
\par
First we must find the ranges for which the two functions are bound. Let us set the equations equal and find the intercepts.
\begin{align}
    x+1            & =x^2-1 \\
    x^2-x-2        & =0     \\
    (x-2)(x+1)     & =0     \\
    \text{Roots: } & 2, -1
\end{align}
Now we must calculate $x$ within the bounds, which is $x+1-(x^2-1) = -x^2+x+2$
\subsection{(a) Square}
\paragraph{Solution} Area of a square is $s^2$. Therefore $A(x)=s^2$. In terms of $x$, this will be $A(x)=(-x^2+x+2)^2$, which I don't want to do. Man I'll do it.
\begin{align}
    (-x^2+x+2)^2 & =(-x^2+x+2)(-x^2+x+2)               \\
                 & = x^4-x^3-2x^2-x^3+x^2+2x-2x^2+2x+4 \\
                 & = x^4-2x^3-3x^2+4x+4                \\
\end{align}
$a=-1$ and $b=2$. Therefore, the integral equation is
\begin{align}
    \int_{-1}^{2}x^4-2x^3-3x^2+4x+4\:dx & =\frac{x^5}{5}-\frac{x^4}{2}-x^3+2x^2+4x\Big|_{-1}^{2} \\
                                        & =(\frac{32}{5})+(\frac{17}{10})                        \\
                                        & =\frac{81}{10}                                         \\
                                        & =\boxed{8.100}
\end{align}
\subsection{(a) Rectangles of Height $1$}
\paragraph{Solution} Area of a rectange with height $1$ is just $s$. Therefore $A(x)=s$. In terms of $x$, this will be $A(x)=-x^2+x+2\:dx$
$a=-1$ and $b=2$. Therefore, the integral equation is
\begin{align}
    \int_{-1}^{2}-x^2+x+2\:dx & =-\frac{x^3}{3}+\frac{x^2}{2}+2x\Big|_{-1}^{2} \\
                              & = (\frac{10}{3})-(-\frac{7}{6})                \\
                              & = \frac{20}{6}+\frac{7}{6}                     \\
                              & = \frac{27}{6}                                 \\
                              & = \boxed{4.5}
\end{align}
\subsection{}

\end{document}