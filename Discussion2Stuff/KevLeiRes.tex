\documentclass[letterpaper, 12pt]{article}
\usepackage[margin=1in]{geometry}
\usepackage{amsmath}
\usepackage{amssymb}
\usepackage{fancyhdr}

\pagestyle{fancy}
\fancyhf{}

\rhead{
    Shendong Li
    Calc 1
}
\rfoot{
    Page \thepage
}

\usepackage{indentfirst}
\setlength{\parindent}{2em}

\begin{document}
\title{Response to Kevin Lei}
\author{by Shengdong Li}
\date{9 April 2020}
\maketitle

\section{Intro}
Hey Kevin, good job on getting to that point in solving the problem! I kept making mistake after mistake, and so this problem took forever to solve. I don't think that the answer key said that your expression was "wrong", it's just different and not simplified to the same degree. You can verify this in two ways, one is just plugging in random values for $r$ and $h$ into the equation of the answer key and your equation and comparing the values, and the other is expanding the $()^3$, and cancelling like terms.

\section{Plugging in Numbers}
\begin{align}
    \intertext{Here we're comparing $\pi(r^2h-\frac{(h-r)^3}{3}-\frac{r^3}{3})$ and $\pi(rh^2-\frac{h^3}{3})$
    }
    \intertext{$r=1$ and $h=1$}
    \pi(r^2h-\frac{(h-r)^3}{3}-\frac{r^3}{3}) & =\pi(1^21-\frac{(1-1)^3}{3}-\frac{1^3}{3})          \\
    \nonumber                                 & =\pi(1-\frac{1}{3})                                 \\
                                              & =\boxed{\frac{2\pi}{3}                            }
\end{align}
\setcounter{equation}{0}
\begin{align}
    \pi(rh^2-\frac{h^3}{3}) & = \pi(11^2-\frac{1^3}{3}) \\
    \nonumber               & = \pi(1-\frac{1}{3})      \\
                            & =\boxed{\frac{2\pi}{3}}
\end{align}
\text{(You can try for other numbers too, I'm sure they work)}
\section{Expanding $()^3$}
\begin{align}
    \intertext{Let's work up to the point where we recieve $()^3$. First integrate...}
    \pi\int_{-r}^{h-r}r^2-x^2\:dx             & = \pi(r^2x-\frac{x^3}{3})\Big|_{-r}^{h-r})                     \\
    \nonumber                                 & = \pi(r^2(h-r)-\frac{(h-r)^3}{3}-(-r^3+\frac{r^3}{3}))         \\
    \intertext{Distribute factors, cancel likes}
    \nonumber                                 & = \pi(r^2h-r^3-\frac{(h-r)^3}{3}+r^3-\frac{r^3}{3})            \\
                                              & = \pi(r^2h-\frac{(h-r)^3}{3}-\frac{r^3}{3})
    \intertext{Now we can expand $\frac{(h-r)^3}{3}$}
    \frac{(h-r)^3}{3}                         & =\frac{1}{3}(h^2-2hr+r^2)(h-r)                                 \\
    \nonumber                                 & =\frac{1}{3}(h^3-2rh^2+r^2h-rh^2+2r^2h-r^3)                    \\
    \nonumber                                 & =\frac{1}{3}(h^3-3rh^2+3r^2h-r^3)                              \\
                                              & = \frac{h^3}{3}-rh^2+r^2h-\frac{r^3}{3}
    \intertext{Plug that back into original expression...}
    \pi(r^2h-\frac{(h-r)^3}{3}-\frac{r^3}{3}) & =\pi(r^2h-\frac{h^3}{3}+rh^2-r^2h+\frac{r^3}{3}-\frac{r^3}{3}) \\
    \intertext{Cancel everything, and we get expression on the answer key}
                                              & = \boxed{\pi(rh^2-\frac{h^3}{3})}
\end{align}
I hope this helped!
\end{document}
