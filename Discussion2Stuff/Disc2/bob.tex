\documentclass[letterpaper,12pt]{article}
\usepackage[margin=1in]{geometry}
\usepackage{amsmath}
\usepackage{amssymb}
%Need this for equations + math
\usepackage{fancyhdr}

\pagestyle{fancy}

\fancyhf{}
\rhead{
    Shengdong Li
    Table 7
    Calc 1
}
\rfoot{Page \thepage}

\begin{document}

\title{Response to Arya Thompson}
\author{by Shengdong Li}
\date{2 April 2020}
\maketitle

\section{Intro}
Hello Arya!
I couldn't really find someone that had a different letter last name than me that didn't already have their work checked, so I'm going to try to do your problem!
\section{Prompt}
\textit{
    \textbf{Last Names (S-Z):}
    Evaluate $\int\sin x\cos^2 x\:dx$, twice. First, use $u = \sin x$ and the fact that $\cos^2 x+\sin^2 x=1$. Then use $u=\cos x$. If you simplify, it's pretty easy to see that these answers are the same, so instead, identify the superior way, and why.
}
\textbf{
    \\~\\ Solution using $u = \sin x$
}
\begin{align}
    \intertext{First setup the equation and $u$ variables}
    \nonumber \int\sin x\cos^2 x\:dx                                                              \\
    \nonumber u                               & =\sin x                                           \\
    du                                        & =\cos x\:dx                                       \\
    \intertext{Then let's rewrite the original expression so that it's easier to slot in our variables later}
    \int\sin x\cos^2 x\:dx                    & =\int\sin x\cos x\cos x\:dx                       \\
    \intertext{Then calculate $\cos x$ using the trig identity}
    \nonumber \cos^2 x+\sin^2 x               & =1                                                \\
    \nonumber \cos^2 x                        & =1-\sin^2 x                                       \\
    \cos x                                    & =\sqrt{1-\sin^2 x}                                \\
    \intertext{Plugin that trig identity to the expression}
    \int\sin                    x\cos^2 x\:dx & =\int\sin x\sqrt{1-\sin^2x}\cos x\:dx             \\
    \intertext{Plugin $du$ and $u$ into the expression}
    \int\sin                    x\cos^2 x\:dx & =\int u\sqrt{1-u^2}\:du
    \intertext{It's hard to continue here, unless we apply a $v$ sub here}
    \nonumber v                               & =1-u^2                                            \\
    \nonumber dv                              & =-2u\:du                                          \\
    -\frac{1}{2}dv                            & =u\:du                                            \\
    \intertext{Now we can plugin $v$ and $dv$ into the $u$ subbed equation}
    \int u\sqrt{1-u^2}\:du                    & =-\frac{1}{2}\int\sqrt{v}\:dv
    \intertext{Then turn the square root into fraction form and integrate}
    \nonumber                                 & =-\frac{1}{2}\int v^{\frac{1}{2}}\:dv             \\
    \nonumber                                 & =-\frac{1}{2}(\frac{2\cdot v^{\frac{3}{2}}}{3})+c \\
                                              & =-\frac{v^{\frac{3}{2}}}{3}+c
    \intertext{Recall the values of $v$ and $u$ to plugin in terms of $x$}
    \nonumber u                               & =\sin x                                           \\
    \nonumber v                               & =1-u^2                                            \\
    \nonumber                                 & =1-\sin^2x                                        \\
    -\frac{v^{\frac{3}{2}}}{3}+c              & =-\frac{(1-\sin^2x)^{\frac{3}{2}}}{3}+c           \\
    \intertext{We can rewrite $1-\sin^2x$ using the original trig identity, which allows for the cancelling of the exponent}
    \nonumber                                 & =-\frac{(\cos^2x)^{\frac{3}{2}}}{3}+c             \\
                                              & =\boxed{-\frac{\cos^3x}{3}+c}
\end{align}

\setcounter{equation}{0}

\textbf{
    \\~\\ Solution using $u = \cos x$
}
\begin{align}
    \intertext{First setup the equation and $u$ variables}
    \nonumber \int\sin x\cos^2 x\:dx                       \\
    \nonumber u            & =\cos x                       \\
    \nonumber du           & =-\sin x\:dx                  \\
    -du                    & =\sin x\:dx                   \\
    \intertext{Plugin $du$ and $u$ into the original equation}
    \int\sin x\cos^2 x\:dx & =-\int u^2\:du                \\
    \intertext{Integrate.}
                           & =-\frac{u^3}{3}+c             \\
    \intertext{Plugin the value of $u$ in terms of $x$}
                           & =\boxed{-\frac{\cos^3x}{3}+c}
\end{align}

\section{Conclusion}
As you can see from the work, solving using $u=\cos x$ is much better than $u=\sin x$. Counting the steps shows that there are almost double the amount of major steps that go into solving the latter vs. the former. Looking at the techniques involved, $u=\sin x$ required you to additionally know a trig identity, recognize when to double substitute variables, and also simplify many times.\\~\\
It looks like we both got the same answers, Arya! I thought that your work had clear steps and helpful comments. Good job on the initial post!



\end{document}
