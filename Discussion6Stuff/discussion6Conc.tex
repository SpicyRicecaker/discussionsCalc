\documentclass[letterpaper, 12pt]{article}
\usepackage[margin=1in]{geometry}
\usepackage{amsmath}
\usepackage{amssymb}
\usepackage{fancyhdr}
\usepackage{hyperref}
\usepackage{xcolor}
\setlength{\headheight}{15pt}

\pagestyle{fancy}
\fancyhf{}

\rhead{
    Shengdong Li
    Calc 1
}
\rfoot{
    Page \thepage
}

\usepackage{indentfirst}

\begin{document}
\title{In Conclusion}
\author{by Shengdong Li}
\date{10 May 2020}
\maketitle

This discussion blew my mind with the amount of different solutions that people had for different problems. \par 
From doing the $12$ problems myself, I reinforced my skill with $u$-sub and different ways to approach problems. \par
Then, from \href{https://bsd.instructure.com/groups/28355/users/33963}{\textcolor{blue}{@Louis Marun}} I was reminded that taking the derivative of a $u$ variable first results in $\frac{du}{dx}$, not $du=dx$, and that could change the way that you look at a problem, in the integral $\int_{ }^{ }\frac{x^{3}}{\sqrt{x^{2}+1}}dx$. \par
From \href{https://bsd.instructure.com/groups/28355/users/26747}{\textcolor{blue}{@Karya Isbara}} I was introduced to the idea of applying polynomial division to help subdivide integrals and thus make problems simpler in solving the intgral $\int_{ }^{ }\frac{x^{2}}{x+4}dx$, which, while it did help break the problem into smaller parts, I learned that it didn't really change the fundamental methods required to solve a problem. \par
From \href{https://bsd.instructure.com/groups/28355/users/51548}{\textcolor{blue}{@Nancy Bai}} I witnessed an amazing way of manipulating the variables in $u$ and $v$ sub in the integral $\int_{ }^{ }x^{3}e^{x^{2}}dx$ that taught me that the initial $u$ value that you generate does not always have to be used, and that you shouldn't be too hasty to immediately plug the $u$ value in, but always solve for $dx$ instead.\par
Finally, from \href{https://bsd.instructure.com/groups/28355/users/26476}{\textcolor{blue}{@Joshua Ji}} I realized a more efficient way to approach the integration of $\int_{ }^{ }\frac{x^{3}}{\sqrt{x^{2}+1}}dx$ with $u=\sqrt{x^{2}+1}$ instead of $u=x^{2}+1$.\par
Overall, I somehow feel a lot more confident with the idea of applying $u$-sub through this discussion, and I'm amazed with the amount of depth that this topic has to offer. 
\end{document}