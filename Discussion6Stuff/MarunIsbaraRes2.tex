\documentclass[letterpaper, 12pt]{article}
\usepackage[margin=1in]{geometry}
\usepackage{amsmath}
\usepackage{amssymb}
\usepackage{fancyhdr}
\setlength{\headheight}{15pt}

\pagestyle{fancy}
\fancyhf{}

\rhead{
    Shengdong Li
    Calc 1
}
\rfoot{
    Page \thepage
}

\usepackage{indentfirst}

\begin{document}
\title{Response to Kayra Isbara}
\author{by Shengdong Li}
\date{8 May 2020}
\maketitle


\section{Intro}
Hey Kayra! Thanks for responding to my initial post. From what I saw, you solved the integral 
$$
    \int_{ }^{ }\frac{x^{2}}{x+4}dx
$$
two times, once using polynomial division into $u$-sub, and once using my method of going straight to $u$-sub.\par
To me, it seemed polynomial division, like regular division, further subdivided the integral, $\int_{ }^{ }\frac{x^{2}}{x+4}dx$, into smaller individual parts, which could then be solved independently. \par
I found it really intersting how at many times it seemed that using polynomial division yielded similar looking integrals to $u$-subbing at the beginning. \par
Maybe that's becauase $u$-sub works in much the same way as polynomial division: it replaces complex parts of an integral with simpler parts that can be solved. \par
In the end however, I agree with your conclusion that in this case using $u$-sub in the beginning was better than polynomial division, because the polynomial division led to $u$-sub anway.\par
\section{Polynomial Division Once}
As a side note, I noticed that you opted to use polynomial divison twice solving this integral, but using it once is also an option. I wonder if dividing once would be an effective method:
\begin{align}
&\int_{ }^{ }\frac{x^{2}}{x+4}dx\\
\intertext{Then long division makes}
&=\int_{ }^{ }x-\frac{4x}{x+4}dx
\intertext{...Which we can then factor out}
&=\int_{ }^{ }xdx-4\int_{ }^{ }\frac{x}{x+4}dx
\intertext{Now let's $u$-sub}
u&=x+4\\
x&=u-4\\
du&=dx
\intertext{Plugin $du$ and $u$}
&=\int_{ }^{ }xdx-4\int_{ }^{ }\frac{u-4}{u}du
\intertext{We can seperate the numerator now}
&=\int_{ }^{ }xdx-4\int_{ }^{ }\frac{u}{u}-\frac{4}{u}du
\intertext{Which simplifies}
&=\int_{ }^{ }xdx-4\int_{ }^{ }1-\frac{4}{u}du
\intertext{Then split into even more integrals}
&=\int_{ }^{ }xdx-4\left(\int_{ }^{ }du-4\int_{ }^{ }\frac{1}{u}du\right)
\intertext{Integrate}
&=\frac{x^{2}}{2}-4x+16-16\ln\left(x+4\right)
\intertext{The constant gets sucked up by the $+C$}
&=\boxed{\frac{x^{2}}{2}-4x-16\ln\left(x+4\right)+C}
\end{align}
\section{Conclusion}
After seeing the integral after each level of polynomial division, I think that you'd agree polynomial division feels like \textbf{it isn't really changing the fundamental required methods to do the problem,} but rather just taking out chunks and pieces of the problem that might make it easier to solve. I think that it's interesting though, and something to keep in mind when attempting $u$-sub. Thanks for replying to my problem!\par
Cheers,\par
Andy Li
\end{document}